\chapter{Conclusion}
\label{sec:conclusion}

Our work has the same goal as many proxy distribution systems; to provide service to clients in a hostile environment with practically an omnipotent censor entity. Our trust-less, elegant approach to proxy distribution relies on essentially hiding \textit{needle} proxies in distribution rounds.

Building trust requires storage of user behaviour over time. Within anonymous systems, the assumption that a system itself can be trusted to store this information is an extremely complex topic, both technically and ideologically. It may be argued that any proxy system provides some measure of anonymity guarantees because many users are aggregated onto single proxies. However, there exists a fundamental distrust of any proxy distribution system that persists user information.

Many trust-based systems are proposed yet few are adopted in practice. Simpler systems such as lightweight proxy distribution and decoy routing are adopted widely, and it may not only be a result of their implementation simplicity. Perhaps, the foundation of trust-based systems goes against the ethos of anonymity. As our privacy concerns grow, we look for more varieties of systems with demonstrable privacy guarantees that align with our privacy needs. The wider the adoption of such systems, the more privacy guarantees we can give. The more honest users in the system, the less of an impact that attackers can have. It requires the effort of everyone to hide vulnerable parties in the crowd. If we volunteer our resources and participate in the privacy community, it is possible.
