\chapter{Conclusion}
\label{sec:conclusion}

My work has the same goal as many trusted proxy distribution systems outlined in the related work in Chapter \ref{sec:related}; to provide service to clients in a hostile environment with practically an omnipotent censor entity. This trust-less, elegant approach to proxy distribution relies on essentially hiding \textit{needle} proxies in distribution rounds.

Building trust requires storage of user statistics over time. Within anonymous systems, the assumption that any user can be trusted to build reputation is an extremely complex topic, both technically and ideologically. While it can be argued that seemingly all proxies provide a measure of anonymity because multiple users are aggregated on a single proxy, there exists a fundamental distrust of any proxy distribution system that persists user information, as it goes against the ethos of anonymity. 

% in addition - prvacy and the different viewpoints of what constitutes privacy, storage of reputation, who to trust to assign trust? - eg. rolling the proverbial diceofproxy assignment