\chapter{Discussion}
\label{sec:discussion}

\section{Future Work}

\textbf{Blocking behaviour.} The simulation can easily extend to model different forms of blocking behaviour by a censor. The parameters for a blocking rate process are coded in the simulation although these were not utilized in the simulation nor analyzed for the evaluation.  

\textbf{Proxies joining and leaving.} A more realistic version of the simulator is to run the experiments with some proxies joining and leaving. This would provide more data on how the needle algorithm preserves new proxies, rather than only dealing with proxies that are created at the same time.

\textbf{Service times.} The simulation was not fully utilized to analyze classic problems like Quality of Service. It would be particularly interesting to examine how non-needle proxies are able to handle the larger loads in a real system.

\textbf{Tor integration.} The needle algorithm could be run inside of the Tor bridedb codebase to validate the simplicity of the algorithm's approach. It would require a different method of client assignment because Tor uses hashes of IP addresses and does not store load or assignment information in the same way that the needle algorithm requires for its operation.